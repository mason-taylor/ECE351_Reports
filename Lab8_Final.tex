

% Default to the notebook output style

    


% Inherit from the specified cell style.




    
        %%%%%%%%%%%%%%%%%%%%%%%%%%%%%%%%%%%%%%%%%%
        % Mason Taylor
        % Signals & Systems Lab
        % Lab #7
        % 10/27/2019
        %%%%%%%%%%%%%%%%%%%%%%%%%%%%%%%%%%%%%%%%%%
    
    \documentclass[12pt]{report}

    
    
        \usepackage[english]{babel}
        \usepackage{url}
        \usepackage[utf8x]{inputenc}
        \usepackage{amsmath}
        \usepackage{graphicx}
        \graphicspath{{images/}}
        \usepackage{parskip}
        \usepackage{fancyhdr}
        \usepackage{vmargin}
        \usepackage{listings}
        \usepackage{hyperref}
        \usepackage{geometry}
        \usepackage{fancyvrb}
        \usepackage{xcolor}
        \usepackage{adjustbox}
        \usepackage{titlesec}
    

    
        \title{Lab \#7}
        \date{10/27/2019}
        \author{Mason Taylor}

        % Setup for chapter titles
        \titleformat{\chapter}
        {\normalfont\LARGE\bfseries}{\thechapter}{1em}{}
        \titlespacing*{\chapter}{0pt}{3.5ex plus 1ex minus .2ex}{2.3ex plus .2ex}

        % Syntax highlighting colors
        \definecolor{codegreen}{rgb}{0,0.6,0}
        \definecolor{codegray}{rgb}{0.5,0.5,0.5}
        \definecolor{codeblue}{rgb}{0,0,0.82}
        \definecolor{codepurple}{rgb}{0.58,0,0.82}
        \definecolor{codeorange}{rgb}{0.75,0.4,0.0}
        \definecolor{backcolour}{rgb}{0.95,0.95,0.92}

        % Commands for Pandoc Conversion
        \providecommand{\tightlist}{%
        \setlength{\itemsep}{0pt}\setlength{\parskip}{0pt}}

        % Colors for the hyperref package
        \definecolor{urlcolor}{rgb}{0,.145,.698}
        \definecolor{linkcolor}{rgb}{.71,0.21,0.01}
        \definecolor{citecolor}{rgb}{.12,.54,.11}

        % ANSI colors
        \definecolor{ansi-black}{HTML}{3E424D}
        \definecolor{ansi-black-intense}{HTML}{282C36}
        \definecolor{ansi-red}{HTML}{E75C58}
        \definecolor{ansi-red-intense}{HTML}{B22B31}
        \definecolor{ansi-green}{HTML}{00A250}
        \definecolor{ansi-green-intense}{HTML}{007427}
        \definecolor{ansi-yellow}{HTML}{DDB62B}
        \definecolor{ansi-yellow-intense}{HTML}{B27D12}
        \definecolor{ansi-blue}{HTML}{208FFB}
        \definecolor{ansi-blue-intense}{HTML}{0065CA}
        \definecolor{ansi-magenta}{HTML}{D160C4}
        \definecolor{ansi-magenta-intense}{HTML}{A03196}
        \definecolor{ansi-cyan}{HTML}{60C6C8}
        \definecolor{ansi-cyan-intense}{HTML}{258F8F}
        \definecolor{ansi-white}{HTML}{C5C1B4}
        \definecolor{ansi-white-intense}{HTML}{A1A6B2}
        \definecolor{ansi-default-inverse-fg}{HTML}{FFFFFF}
        \definecolor{ansi-default-inverse-bg}{HTML}{000000}

        % Exact colors from NB
        \definecolor{incolor}{rgb}{0.0, 0.0, 0.5}
        \definecolor{outcolor}{rgb}{0.545, 0.0, 0.0}

        %Listing colors
        \lstdefinestyle{mystyle}{
            backgroundcolor=\color{backcolour},   
            commentstyle=\itshape\color{codegray},
            keywordstyle=\color{codegreen},
            numberstyle=\tiny\color{codepurple},
            stringstyle=\color{codeorange},
            identifierstyle=\color{codeblue},
            basicstyle=\ttfamily\footnotesize,
            breakatwhitespace=false,         
            breaklines=true,                 
            captionpos=b,                    
            keepspaces=true,                 
            numbers=none,                    
            numbersep=5pt,                  
            showspaces=false,                
            showstringspaces=false,
            showtabs=false,                  
            tabsize=2
        }
        \lstset{style=mystyle}

        % Define a nice break command that doesn't care if a line doesn't already exist.
        \def\br{\hspace*{\fill} \\* }
        % Math Jax compatibility definitions
        \def\gt{>}
        \def\lt{<}
        \let\Oldtex\TeX
        \let\Oldlatex\LaTeX
        \renewcommand{\TeX}{\textrm{\Oldtex}}
        \renewcommand{\LaTeX}{\textrm{\Oldlatex}}
    

    
        % Prevent overflowing lines due to hard-to-break entities
        \sloppy 
        % Setup hyperref package
        \hypersetup{
        breaklinks=true,  % so long urls are correctly broken across lines
        colorlinks=true,
        urlcolor=urlcolor,
        linkcolor=linkcolor,
        citecolor=citecolor,
        }
        \makeatletter
        \let\thetitle\@title
        \let\reptitle\@title
        \let\theauthor\@author
        \let\thedate\@date
        \makeatother
        \pagestyle{fancy}
        \fancyhf{}
        \rhead{\theauthor}
        \lhead{\reptitle}
        \cfoot{\thepage}
        
            \setmarginsrb{3 cm}{2.5 cm}{3 cm}{2.5 cm}{1 cm}{1.5 cm}{1 cm}{1.5 cm}
        
    


    \begin{document}
    
    
\begin{titlepage}
	\centering
\begin{center}    
    \textsc{\Large Signals \& Systems Lab}\\[2.0 cm]	
    \textsc{\Large ECE 351 - 51}\\[0.5 cm]
    \textsc{\small Instructor: Dr. Dennis Sullivan}\\[0.5 cm]
    \textsc{\small Lab TA: Phillip Hagen}
\end{center}
	\rule{\linewidth}{0.4 mm} \\[0.4 cm]
	{ \huge \bfseries \thetitle}\\
	\rule{\linewidth}{0.4 mm} \\[1.5 cm]
	
	\begin{minipage}{0.4\textwidth}
		\begin{flushleft} \large
		%	\emph{Submitted To:}\\
		%	Name\\
          % Affiliation\\
           %contact info\\
			\end{flushleft}
			\end{minipage}~
			\begin{minipage}{0.4\textwidth}
            
			\begin{flushright} \large
			\emph{Prepared By :} \\
			\theauthor  \\
			\thedate
		\end{flushright}  
	\end{minipage}\\[2 cm]
\end{titlepage}
    \tableofcontents
    \pagebreak


    
\hypertarget{introduction}{%
\chapter{Introduction}\label{introduction}}

In this lab we will be using Fourier series to approximate signals in
the time domain. To do this, we will solve for the Fourier series
coefficients formulas and use Python to do the calculations of the
numerical coefficients. \# Equations \#\# Part 1 - Task 1 We will be
plotting the Fourier series for various N (the upper limit of the
summation), so first we need the formula for the coefficients. The plot
we will be analyzing is:

\begin{figure}[h]
\centering
 \includegraphics[width=0.8\textwidth]{signal.png}
 \caption{The signal we will approximate with Fourier Series.}
\end{figure}

\br  We solve the following to get the coefficients:
\[ a_k=\int_{0}^{T} x(t)cos(k\omega_0t) dt \]
\[ b_k=\int_{0}^{T} x(t)sin(k\omega_0t) dt \] where:
\[ \omega_0=\frac{2\pi}{T} \] Therefore: \[ a_k = 0 \]
\[ b_k = \frac{2}{k\pi}\left(1-cos(k\pi)\right) \]

Note that all \(a_k\) coefficients are 0 since there is no DC term
(k=0), and the signal is orthogonal to cosine and thus is a pure sine
series (since it is a perfect odd function). We input these coefficient
functions into Python and we will use them in Task 2 to create the
plots.

\hypertarget{procedure}{%
\chapter{Procedure}\label{procedure}}

\hypertarget{part-1}{%
\section{Part 1}\label{part-1}}

In this part, we will calculate some of the values of the Fourier
expansion of the function given, and use these values to plot the
approximation of the function.

\hypertarget{task-1}{%
\subsection{Task 1}\label{task-1}}

We have calculated the first several values of \(a_k\) and \(b_k\).
These can be seen in Appendix I.

\hypertarget{task-2}{%
\subsection{Task 2}\label{task-2}}

We now plot the Fourier series expansion for various N values (the
number of terms calculated). We will view the plots for N = \{1, 3, 15,
50, 150, 1500\} with an 8 second period, and 20 second plot range. The
plots are performed in 2 different ways, one method quickly plots all
the levels of N in a single plot, to have a colorful visual on how each
individual term adds to the plot. The second method divides each level
of N into it's own plot resulting in 2 figures with 3 plots each.

    \begin{figure}
    \begin{center}
    \adjustimage{max size={\linewidth}}{output_4_0.png}
    \end{center}
    \caption{A plot of the six different levels of summation for the approximation of the square wave.}
\end{figure}
    
    \begin{figure}
    \begin{center}
    \adjustimage{max size={\linewidth}}{output_5_0.png}
    \end{center}
    \caption{The different levels of approximation broke into six seperate plots.}
\end{figure}
    
    \begin{figure}
    \begin{center}
    \adjustimage{max size={\linewidth}}{output_5_1.png}
    \end{center}
    \caption{The different levels of approximation broke into six seperate plots.}
\end{figure}
    
\hypertarget{questions}{%
\chapter{Questions}\label{questions}}

\begin{enumerate}
\def\labelenumi{\arabic{enumi}.}
\tightlist
\item
  \textbf{Is \(x(t)\) an even or an odd function? Explain why.}
  \br \(x(t)\) is an odd function since it satisfies the condition that
  \(x(-t)=-x(t)\). This means that the function is reflected through the
  origin.
\item
  \textbf{Based on your results from Task 1, what do you expect the
  values of a2, a3, . . . , an to be? Why?} \br The values of \(a_n\)
  for all n will be 0 since the function is odd, meaning that the
  integral of the inner product with cosine will end up being 0.
\item
  \textbf{How does the approximation of the square wave change as the
  value of N increases? In what way does the Fourier series struggle to
  approximate the square wave?} \br The approximation starts to approach
  the actual shape of the square wave with a steeper rise and less
  oscillation in the peak and trough values. As the number of terms
  increases, more of the fundamental frequencies are added back making
  the approximation more accurate.
\item
  \textbf{What is occurring mathematically in the Fourier series
  summation as the value of N increases?} \br As N is increased, we add
  more terms with different frequencies and each term is multiplied by a
  certain magnitude representing how much of that frequency is apparent
  in the original function. Each new sine term takes parts away that do
  not contribute to the signal and adds parts that do contribute to the
  signal. Thus, the more frequencies we include, the closer the
  approximation is to the original signal. This can be seen in the final
  plot at N = 1500, the signal looks nearly identical to the original
  square wave.
\end{enumerate}

\hypertarget{conclusion}{%
\chapter{Conclusion}\label{conclusion}}

In this lab we have investigated the Fourier series approximation of a
square wave signal. The Fourier series provides an approximation for any
periodic signal in terms of only sines and cosines with different
magnitudes and frequencies. This can be very useful in analyzing signals
to determine what frequencies contribute to the signal. This also allows
us to isolate only the frequencies that we are interested in, to reduce
high frequency noise for example, and extract a clean approximation of
the signal we want. The Fourier series and transforms have wide reaching
applicability especially in electronics and communications technology.

\hypertarget{appendix-i-coefficient-values}{%
\chapter{Appendix I: Coefficient
Values}\label{appendix-i-coefficient-values}}

Below are the results of calculating the first several coefficient
values using the formulas found earlier:

    \begin{Verbatim}[commandchars=\\\{\}]

a[ 0 ] =  0.0

b[ 0 ] =  0.0

a[ 1 ] =  0.0

b[ 1 ] =  1.2732395447351628

a[ 2 ] =  0.0

b[ 2 ] =  0.0

a[ 3 ] =  0.0

b[ 3 ] =  0.4244131815783876

    \end{Verbatim}

\hypertarget{appendix-ii-multiple-approaches-to-plotting}{%
\chapter{Appendix II: Multiple Approaches to
Plotting}\label{appendix-ii-multiple-approaches-to-plotting}}

There are certainly multiple ways to implement automatic plotting of
several functions at once. The simplest, is plotting all functions in
one graph. This is implemented by the code in Listing 1. The advantage
of this approach is that, in this case, it is able to reuse the work in
creating lower order plots to create the higher order plots, rather than
creating each plot from scratch.

\begin{lstlisting}[language=Python, caption=Python implementation to generate a graph with all the plots on one axis.]
T = 8
N1 = [1, 3, 15, 50, 150, 1500]
t = np.arange(0, 20 + stepsize, stepsize)
data = []

data = np.zeros(len(t))
plt.subplot(1,1,1)
plt.title
# For every k up to the max we need, add the contributions at 
# that frequency
for k in range(max(N1)):
    for t_val in range(len(t)):
        data[t_val] += bk[k] * np.sin(2*np.pi*k*t[t_val]/T)
    # If we are at a desired value of k, we plot the current results on 
    # the existing plot.
    if k in N1:
        plt.plot(t,data)

\end{lstlisting}

Another approach, is to adapt this method to split the plots out when we
need to. This method is implemented in Listing 2. It simply keeps count
of the plots and makes a new figure every 3 plots, keeping each plot in
its' own subfigure slot utilizing the modulus operator. This keeps the
speed advantage of the previous method while also organizing the plots
and making them easy to understand. \pagebreak 

\begin{lstlisting}[language=Python, caption=Adapted implementation to give each plot its own subplot.]
pltcount = 0
data = np.zeros(len(t))
plt.figure(figsize=(10,10))
plt.title('Fourier Series Summation for Various N')
for k in range(max(N1)+1):
    for t_val in range(len(t)):
        data[t_val] += bk[k] * np.sin(2*np.pi*k*t[t_val]/T)
    if k in N1:
        plt.subplot(3, 1, pltcount % 3 + 1)
        if (pltcount % 3) == 0:
            plt.title('Fourier Series Summation for Various N')
        plt.grid(True)
        plt.plot(t,data)
        plt.ylabel('Plot for N=' + str(k))
        pltcount += 1
        if (pltcount % 3) == 0 or pltcount == len(N1):
            plt.show()
            if pltcount < len(N1):
                plt.figure(figsize=(10,10))

\end{lstlisting}


    % Add a bibliography block to the postdoc
    
    
    
    \end{document}
